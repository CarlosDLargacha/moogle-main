\documentclass{article}
\usepackage{amsfonts}
\usepackage{amsmath}
\usepackage{graphicx}
\usepackage[margin=2cm]{geometry}

\begin{document}

\title{\textbf{Proyecto de Programación I \\Moogle!}}
\author{Carlos Daniel Largacha Leal C-112}
\date{}

\maketitle

\vspace{10pt}

\section* {Ejecución del Moogle!}

\begin{itemize}
    \item  Copiar a la carpeta “Content” los documentos
    \item  Asegurarse de que los documentos sean de extensión “.txt”
    \item  Ejecutar el programa
\end{itemize}


\vspace{10pt}

\section* {¿Qué hace el programa?}
Al introducir una cadena de texto en el cuadro de búsqueda y hacer clic en el botón 
buscar Moogle! buscará la cadena de texto introducida en el conjunto de 
documentos dado, teniendo en cuenta la relación de cada palabra en cada uno de 
los documentos. 
Se mostrarán los resultados más relevantes para su búsqueda y en cada uno se 
mostrará una sección del texto y el nombre del documento en el que se encontrará 
la cadena de texto introducida.
Ante una consulta con un error de escritura, el programa sugerirá una cadena de 
texto similar a la introducida que se encuentre en los documentos. Para usarla copie 
la sugerencia en el cuadro de diálogo y volver a realizar la búsqueda. 
\vspace{10pt}

\section* {El código está agrupado bajo el namespace MoogleEngine y 
consta de 7 clases:}

\begin{itemize}
    \item  Moogle: clase principal del proyecto.
    \item  Ordenar: clase encargada de ordenar los resultados de búsqueda.
    \item  Searchitem: información acerca de los documentos que responden a la búsqueda.
    \item  Searchresult: representa el resultado de la búsqueda
    \item  Suggestion: provee una sugerencia a partir de la consulta del usuario
    \item  TF$-$IDF: convierte el conjunto de documentos en una matriz de vectores
    \item  Utilidades: multiplica una lista de números decimales (donde se encuentra el tf$-$idf 
de la consulta) y se multiplica por una lista de matrices.   
\end{itemize}


\vspace{10pt}

\section* {¿Cómo funciona el programa?}
Comenzando con la clase (TFIDF), esta se encarga de crear un diccionario que 
relaciona las diferentes palabras de todos los archivos de texto con su orden de 
aparición. Luego, se genera una lista de matrices, donde cada posición "n" de la lista 
corresponde a la posición o valor de la palabra "n" en el diccionario. De esta 
manera, las matrices tienen un tamaño igual a la cantidad de archivos de texto. El 
objetivo de esta lista es organizar el tf-idf de cada palabra en cada texto.
También se tiene una lista (vector$-$idf) en la que se almacenan los valores idf de 
cada palabra y una lista de cadenas (títulos) donde se guardan los títulos de cada 
archivo de texto.

\vspace{10pt}

\section*{¿Qué es tf-idf?}
tf-idf (Term frequency – Inverse document frequency), es una medida numérica que 
expresa cuán relevante es una palabra para un documento en una colección. El 
valor tf-idf aumenta proporcionalmente al número de veces que una palabra 
aparece en el documento, pero es compensada por la frecuencia de la palabra en 
la colección de documentos, lo que permite manejar el hecho de que algunas 
palabras son generalmente más comunes que otras.\\ \\ El tf$-$idf es producto de dos medidas:

\vspace{10pt}

 $tf-idf(t,d,D) = tf(t,d) x idf(t,D)$\\ \\tf(t,d) representa la frecuencia de una palabra t en un documento d, es decir la 
cantidad de veces que una palabra se repite en un documento.\\ \\Idf(t,D) es frecuencia inversa de documento, es una medida de si la palabra t es 
común o no, en la colección de documentos. Se obtiene dividiendo el número total 
de documentos por el número de documentos que contienen el término, y se toma 
el logaritmo de ese cociente:

\vspace{10pt}

$idf(D,t) = log\dfrac{D}{d(t)}$

\vspace{10pt}

En la clase Utilidades, se encuentran dos métodos: uno que multiplica una lista de 
números decimales (donde se encuentra el tf-idf de la consulta) por la lista de 
matrices, y otro que realiza una multiplicación elemento a elemento entre una 
matriz y una lista de igual dimensión.

\vspace{10pt}

En la clase Moogle, la consulta se convierte en una matriz de cadenas separadas 
por palabras y luego se convierte en un vector calculando su tf. Este vector se 
multiplica término a término con el vector que contiene los valores idf de los textos, 
y con el vector resultante se calcula el puntaje de cada texto, multiplicándolo por 
la matriz que contiene los tf-idf de todos los textos. Se busca el fragmento relevante 
revisando cada texto relevante y buscando la palabra más relevante de ese texto 
para la búsqueda, y se devuelve una cadena de 20 palabras después de esa palabra.

\vspace{10pt}

Finalmente, se crea el objeto SearchResult y se ordena según su puntaje con la 
función Sort de la clase Ordenar, y se devuelve este objeto. 

\vspace{10pt}

La clase Suggestion sirve para en caso de que alguna de las palabras de la consulta 
no esté presentes en ninguno de los documentos, se sugerirá una palabra que si 
esté presente en estos. Esto se hace empleando el algoritmo de Levenshtein: el 
cual permite calcular la cantidad de transformaciones mínimas a la que debe ser 
sometida una palabra para transformarse en otra. Estas transformaciones son:

\begin{itemize}
	\item  Inserción: agregar un carácter.
	\item  Eliminación: quitar un carácter
	\item  Sustitución: cambiar un carácter por otro.
\end{itemize}

\end{document}
